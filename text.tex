\section{Notation}

For the rest of this course, $\N$ contains 0.

\section{Introduction}

\begin{definition}
  Unless stated otherwise, $G \ss \C$ and $H \ss \R^n$ are arbitrary domains.
\end{definition}

\begin{definition}
  A function $f \col G \to \C$ is \emph{analytic}, if for any $z_0 \in G$ there exists $r > 0$ such that $D_r(z_0) \ss G$, and
  $$ f(z) = \sum_{n \in \N_0} a_n (z - z_0)^n $$
  for some $\b{a_n}$ and every $z \in \D_r(z_0)$. 
\end{definition}

\begin{definition}
  $\phi \col G \to \C$ is \emph{holomorphic} at $z_0$, iff exists
  $$ \phi'(z_0) = \lim_{h \to 0} \frac{\phi(z_0+h)-\phi(z_0)}{h}. $$
  Here $h \in \C$.
  If a function $f$ is holomorphic at every point of $E \ss \C$, we write $f \in \Hol E$.
\end{definition}

\begin{definition}
  A function $f \in \Hol \C$ is called \emph{entire}.
\end{definition}

\begin{theorem}[Cauchy-Riemann equations]
  $f \col G \to \C$ is holomorphic at $z_0$ iff
  \begin{align*}
    \frac{\pd u}{\pd x} &= -\frac{\pd v}{\pd y}, x  x x&
    \frac{\pd u}{\pd y} &= \frac{\pd v}{\pd x}
  \end{align*}
  at $\p{x_0, y_0}$,
  where $u(x, y)+iv(x,y) = f(x+iy)$ and $z_0 = x_0+iy_0$.
\end{theorem}
That is, the Jacobi matrix of $u \times v$ is in the image of the standard embedding of $\C$ into $\on{M}_2(\R)$:
$$ a+ib \mapsto \begin{pmatrix}
  a & b \\
  b & -a
\end{pmatrix}
$$

\begin{proof}[Has been proven in semester II.]
\end{proof}

\begin{lemma}
  If $f \col G \to \C$ is analytic, then it is holomorphic.
\end{lemma}

\begin{proof}
  The series for $f(z)$ can be differentiated.
\end{proof}

\begin{theorem}[Cauchy]
  Let $f \col G \to \C$ be holomorphic.
  Then it is analytic at every $x_0 \in G$, with the radius of convergence $r$ being equal to $r = \dist\p{x_0, \C \sm G}$. 
\end{theorem}

The proof will be given shortly.

\section{Differential forms}

\subsection{A reminder}

\begin{definition}
  If we have a form
  $$ \go(h) = \sum_I \go_I \cdot h^I, $$
  its integral (also a form) is defined as
  $$ \int \go = \int \sum_I \go_I \circ x \cdot D_I, $$
  where $D_I$ is the determinant of the rows $I$ of the Jacobi matrix $\d x$.
\end{definition}

\subsection{Integral of a form along a path}

\begin{definition}[integral along a curve] 
  Let $\gg \col [a, b] \to \R^n$ be a $C^1$ function. If $\phi = f_1 \d x_1 + \dots + f_n \d x_n$, where $f_i$ are continuous complex functions on $G$, then
$$ \int_\gg \phi := \sum_{j=1}^n\ \int_{t=a}^b f_j\p{\gg(t)} \gg'_j(t). $$
\end{definition}

Evidently, the integral over a one-dimensional submanifold does not depend on parametrisation. We will further use that to write integrals over subsets of $\C$, not curves.

\begin{remark}
  We may only require that $\gg$ is rectifiable.
  In this case, the integral will be in the sense of Stieltjes:
  $$ \int_\gg \Phi = \sum_{j=1}^n \int_{t=a}^b f_j\p{\gg(t)} \dd \gg_j. $$
  We will not need this during this course.
\end{remark}

\begin{lemma}
  Integral of differential forms along a path is linear with respect to the form.
\end{lemma}

\begin{proof}
  Evident.
\end{proof}

\begin{lemma}[change of variables]
  Let $\ga \col [c, d] \to [a, b]$ be a $C^1$-homeomorphism, and $\wt \gg = \gg \circ \ga$. Then
  $$ \int_{\wt \gg} \phi = \pm \sum_{j=1}^n \ \int_{t=c}^d f_j \circ \gg \circ \ga(s) \cdot \gg'_j\circ\ga(s) \cdot \ga'(s). $$
  The sign here depends on whether $\ga$ is increasing or decreasing.
\end{lemma}

\begin{proof}
  Follows from the change-of-variables formula for the Riemann integral.
\end{proof}

\begin{definition}
  Let $\ga, \gb$ be $C^1$-paths. Their \emph{concatenation} $\ga \gb$ is defined as
  $$ \gg(t) =
  \begin{cases}
    \ga(t), & t \in [a, b], \\
    \gb \circ \phi(t), & t \in [b, c],
  \end{cases}
  $$
  where $\phi \col [b, c] \to [a', b']$ is a homeomorphism.
\end{definition}

\begin{definition}
  A path $\gg$ is \emph{piecewise smooth}, iff it is a finite concatenation of smooth paths.
\end{definition}

\begin{definition}
  The integral of a form along a piecewise smooth path is the sum of integrals over its components.
\end{definition}

\begin{definition}
  If $\phi = \sum \phi_j \d x_j$ is a differential form, we denote
  $$ \n{\phi} = \sqrt{\sum_{j=1}^n \phi_j^2}. $$
\end{definition}

Differential 1-forms are simply functions between Euclidean spaces.

\begin{theorem}[principal estimate]
  If $\gg$ is piecewise smooth and $\phi$ is a continuous differential 1-form in a neighbourhood of $\im \gg$, then
  $$ \a{\ \int_\gg \phi\ } \le l(\gg) \cdot \sup_{x \in \im \gg} \n{\phi(x)}. $$
\end{theorem}

\begin{idea}
  CBS.
\end{idea}

\section{Antiderivatives}

\begin{definition}
  Let $\go$ be be a differential 1-form in $G$.
  Its \emph{derivative} is the form
  $$ \d \go = \sum_{j=1}^n \frac{\pd \go_j}{\pd x} \d x_j. $$
\end{definition}

\begin{definition}
  Let $G \ss \R^n$ be a differential form.
  $F \col G \to \C$ is called the \emph{antiderivative} of $\Phi$, iff $\d F = \Phi$. 
\end{definition}

\begin{definition}
  A differential form $\go$ is
  \begin{enumerate}
    \item \emph{exact}, iff it has an antiderivative;
    \item \emph{closed}, iff every point $x \in G$ has a neighbourhood where $\go$ is exact.
  \end{enumerate}
\end{definition}

  Observe that this definition differs from the one given in the semester III. This one is more general: the previous one depended on smoothness.

\begin{lemma}
  Suppose $\go$ is a $C^1$ differential 1-form in $G$.
  Then $\go$ is closed iff
  $$ \pd_i \go_j = \pd_j \go_i $$
  for all $i, j \in \b{1, \dots, n}$.
\end{lemma}

\begin{proof}[Proof of $\Rightarrow$.]
  Locally, we have an antiderivative $\Omega$, so $\pd_j \Omega = \go_j$.
  Then
  $$ \pd_i \go_j = \pd_{i} \pd_j \Omega = \pd_j \pd_i \Omega = \pd_j \go_i. $$
\end{proof}

\begin{proof}[Proof of $\Leftarrow$.]
  We know from semester III that every differential form $\go$ such that $\d \go = 0$ is exact. But this is true of $\go$.
\end{proof}

\begin{lemma}
  Let $\gg$ be a piecewise smooth path with $\im \gg \ss G$ and ends $A, B$. Then
  \begin{equation}
    \label{NL}
    \int_\gg \d F = F(B) - F(A).
  \end{equation}
\end{lemma}

\begin{proof}
  From the Newton-Leibniz formula.
\end{proof}

\begin{theorem}
  Every two points in $G$ can be connected by piecewise linear path.  
\end{theorem}

\begin{proof}[A well-known fact.]
\end{proof}

\begin{definition}
  Let $\Phi$ be a differential form in a region $H \ss \R^n$.
  We call $H$ a \emph{$\Phi$-balanced}\footnote{My own term.} region, iff
  $ \int_\gg \Phi = 0 $
  for every closed curve $\gg$ with $\im \gg \ss H$.
\end{definition}

\begin{theorem}[reformulations of `exact']
  Let $\Phi$ be a differential form in $H \ss \R^n$ with continuous coefficients. Equivalent are:
  \begin{enumerate}
    \item $\int_\gg \Phi$ depends on $A$ and $B$ only.
    \item $H$ is $\Phi$-balanced.
    \item $\Phi$ is exact. 
  \end{enumerate} 
\end{theorem}

\begin{proof}[Proof of $3 \Rightarrow 2$.]
  See \eqref{NL}.
\end{proof}

\begin{proof}[Proof of $2 \Rightarrow 1$.]
  Concatenate paths between two points and get a closed path. But the integral (which is zero by hypothesis) splits into two. Next we use that it depends on the direction of the path.
\end{proof}

\begin{proof}[Proof of $1 \Rightarrow 3$.]
  Fix $A_0 \in G$. For any $x \in H$, let $\gg$ be a piecewise linear path $A_0 \leadsto x$. Define
  $$ F(x) = \int_\gg \Phi. $$
  Since the integral depends only on $x$, this is correctly defined function
  We assert that $F$ is an antiderivative for $\Phi$; that is, the partial derivatives of $F$ are components of $\Phi$.
  To see this, consider the path
  $$ A_0 \leadsto_\gg x \leadsto_\gb x+te_j, $$
  where the last part $\gb$ is linear.
  Then
  \begin{align*}
    \frac{F(x+te_j) - F(x)}{t}
    &= \frac{\int_\gg \Phi + \int_{\gb} \Phi - \int_\gg \Phi}{t}  \\
    &= \frac{1}{t} \int_{\gb} \Phi \\
    &= \frac{1}{t} { \sum_{k=1}^n\ \int_{\gt=0}^t f_k\p{x+\gt e_j} \gb_k'(\gt) } \\
    &= \frac{1}{t} \int_{\gt=0}^t f_j\p{x+\gt e_j} \\
    &\xrightarrow[t \to 0]{} {f_j(x)}.
  \end{align*}
\end{proof}

\begin{definition}
  We call a region $H \ss \R^2$ \emph{rectangle-astroid}\footnote{My own term.}, iff there exists $x_0 \in H$ such that for every $x \in H$ the 2-dimensional rectangle with sides parallel to the axes, having $x_0$ and $x$ as diametral points, lies in $H$ together with its closure. The rectangle in this context is called \emph{central}.
\end{definition}

This definition is long, but it highlights what we need to use to prove the following proposition for circles.

\begin{lemma}[addition to the theorem]
  Let $H \ss \R^2$ be a rectangle-astroid region. Then $H$ being $\Phi$-balanced is also equivalent to
  $$ \int_Q \Phi = 0 $$
  for every 1-dimensional central rectangle $Q$. 
\end{lemma}

\begin{idea}
  The integral over this rectangle is equal to zero, since it is closed. Conversely, we can find an antiderivative like in the proof of the theorem.
\end{idea}

\begin{theorem}[Cauchy, Morera]
  Let $\gg$ be a closed continuous curve with $\im \gg \ss G$.
  Then a function $f \col G \to \C$ is holomorphic iff
  $$ \int_\gg f = 0. $$
\end{theorem}

\section{Connectivity}

\begin{lemma}
  Every two points of a domain $H \ss \R^n$ can be connected by a piecewise-linear path.  
\end{lemma}

\begin{proof}
  Fix $x_0 \in H$ and put $A$ to be the set of all points reachable from $x_0$ by a piecewise-linear path over $H$. $A$ is open, since $H$ is: every point in $A$ is the centre of a ball in $H$, and balls are convex. The complement $H \sm A$ is open for the same reasons: take any point $x$ from there, there is a ball $B \ss H$ around it; if there was a point of $A$ in this $B$, we could connect it to $x$. Therefore, $A$ is open and closed; but it is not empty either, so $A = H$ by connectedness of the domain $H$.
\end{proof}

Recall the theorem on equivalence of linear connectivity and connectedness for locally linearly connected spaces. The proof is almost the same.

\begin{theorem}
  Every two points of a domain $H \ss \R^n$ can be connected by a $C^\infty$ path.   
\end{theorem}

We need to recall some stuff about convolutions for the proof.

\begin{definition}
  A family $\b{\phi_s}_{s > 0}$ of infinitely smooth functions $\R^n \to \R$ with compact support and such that
  \begin{align*}
    \int \phi_s &= 1, &
    \phi_s(x) &= \frac{\phi_1\p{x/s}}{s^n}
  \end{align*}
  for all $s > 0$ and $x \in \R^n$ is called a
  \emph{standard approximative unit} or a \emph{mollifier}.
\end{definition}

\begin{theorem}
  ~\begin{enumerate}
    \item A family $\b{\phi_\square}$ exists.
    \item Let $f \col \R^n \to \R$ be a continuous function.
  Then the functions $\phi_s * f$ converge to $f$ uniformly with $s \to 0$ and are infinitely smooth.
  \end{enumerate}
\end{theorem}

The \emph{proof} was given in the third semester.
Now we start with the main theorem.

\begin{proof}
  Let $\gg \col [a, b] \to H$ be a continuous path $A \leadsto B$. Continue $\gg$ to a continuous path $\R \to H$ by setting
  $$
  \gg(t) =
  \begin{cases}
    \gg(a), & t < a, \\
    \gg(b), & t > b, \\
    \gg(t), & t \in [a, b].
  \end{cases}
  $$
  Let $\phi_\square$ be a standard approximative unit for functions on $\R$, and put for all $s > 0$
  $$ \wh \gg_s(x) = \p{\phi_s*\gg}(x), $$
  where $*$ denotes component-wise convolution.
  Fix $\eps > 0$, $a_2 < a$, and $b_2 > b$.
  By the theorem on approximating with units, if $s$ is small enough, we have $\a{\wh \gg_s(t) - \gg(t)} < \eps$  for all $t \in [a_2, b_2]$.
  This $\eps$ can be chosen in such a way that the path $\wh \gg_s$ does not leave the domain $H$.
  Further, choose $a_2, b_2$ such that
  \begin{align*}
    \wh \gg_s\p{a_2} &= \gg(a), &
    \wh \gg_s\p{b_2} &= \gg(b).
  \end{align*}
  \wtf
  This ensures  $\wh \gg|_{\q{a_2, b_2}}$ is indeed the path we need.
\end{proof}

\section{Closed forms and balanced regions}

\begin{theorem}[reformulations of `closed']
  Let $\Phi$ be a differential form in $H \ss \R^n$ with continuous coefficients $f_j$. The following are equivalent:
  \begin{enumerate}
    \item $\Phi$ is closed.
    \item Every $x \in H$ has a $\Phi$-balanced neighbourhood $U \ss H$.
  \end{enumerate}
  In case $n = 2$, two more reformulations are true:
  \begin{enumerate}
    \setcounter{enumi}{2}
    \item For every $x \in H$, there exists a rectangle-astroid region $B \ni x$ such that $\int_Q \Phi = 0$ for any central rectangle $Q$.
    \item $\int_Q \Phi = 0$ for any rectangle $Q$ such that $\Cl Q \ss H$. 
  \end{enumerate}
\end{theorem}

\begin{proof}[Proof of $3 \Rightarrow 4$.]
  Split the rectangle $Q$ into equal four, $\b{Q_i}$.
  Then the integral over $Q$ is equal to the sum of integrals over $\b{Q_i}$.
  Cover the compact $\Cl Q$ with `good' rectangle-astroid regions which lie in $H$ completely (they exist by hypothesis).
  Let $\gd > 0$ be the Lebesgue number of this cover.
  Continue to split the rectangles into four until each of them is less than $\gd$ in diameter.
  Now it is clear that the integral over $Q$ is zero itself.
\end{proof}

\begin{proof}[Proof of $4 \Rightarrow 3$]
  $G$ is open.
\end{proof}

\section{Change of basis}

Let $z = x+iy$. Then
\begin{align*}
  \d z &= \d x + i \d y &
  \d \ol z &= \d x + i \d y,
\end{align*}
so
\begin{align*}
  \d x &= \frac{\d z + \d \ol z}{2}, &
  \d y &= \frac{\d z - \d \ol z}{2}.
\end{align*}
Now let $\phi = u \d{x} +v\d{y}$ be a 1-form, defined in $H \ss \R^2$. Then, if $\phi = \d{\Phi}$ for a function $\Phi \col H \to \C$, we have
\begin{align*}
  \Phi &= \frac{\pd_1 \Phi - \pd_2 \Phi}{2} \d z
  + \frac{\pd_1 \Phi - \pd_2 \Phi}{2} \d \ol z
\end{align*}
by direct computation.
By analogy, we define
\begin{definition}
  \begin{align*}
  \pd_z \Phi &:= \frac{\pd_1 \Phi - i \pd_2 \Phi}{2} &
  \pd_{\ol z} \Phi &:= \frac{\pd_1 \Phi + i \pd_2 \Phi}{2}.
\end{align*}
\end{definition}
Let $\Phi = p + iq$.
It then can be derived that
\begin{align*}
  \pd_{\ol z} \Phi
  &= \frac{\pd_1 p - \pd_2 q}{2}
  + i \frac{\pd_1 q + \pd_2 p}{2}.
\end{align*}
Therefore,
\begin{align*}
  \pd_{\ol z} \Phi = 0 
  \iff
  \begin{cases}
    \pd_1 p = \pd_2 q, \\
    \pd_1 q = -\pd_2 p.
  \end{cases}
\end{align*}
These are the Cauchy-Riemann equations.
Therefore,
\begin{lemma}
  A function $\Phi \col G \to \C$ is holomorphic iff $\d{\Phi} = \pd_{z} \Phi \d{z}$.
\end{lemma}

We can also prove the following:
\begin{lemma}
  Let $\ga \d z$ be a form in $G$.
  It is exact iff there exists a holomorphic function $A \col G \to \C$ such that $A' = \ga$.
\end{lemma}

\begin{proof}[Proof of $\Rightarrow$.]
  Let $A$ be the antiderivative.
  $\ga \d{z} = \ga\d{x}+i\ga\d{y}$.
  Then $\pd_1 A = \ga$, $\pd_2 A = i\ga$,
  so
  \begin{align*}
    A' = \pd_z A &= \frac{\pd_1 A - i\pd_2 A}{2} = \ga, &
    \pd_{\ol z} A &= \frac{\pd_1 A + i\pd_2 A}{2} = 0.
  \end{align*}
\end{proof}

\begin{lemma}[a variation of the principal estimate]
  Let $\ga \col G \to \C$ be a function, $\gg \col [a, b] \to G$ a $C^1$ path.
  Then
  $$ \a{\int_\gg \ga \d{z}} \le l(\gg) \cdot \sup_{z \in \im \gg} \a{\ga(z)}. $$
\end{lemma}

\begin{idea}
  Use the principal estimate and the identity $\gg' = \gg'_1 + i \gg'_2$.
\end{idea}

\section{Cauchy's theorem on closedness}

\begin{theorem}[Cauchy, on closedness]
  If $f \col G \to \C$ is holomorphic, then the form $f \d{z}$ is closed.
\end{theorem}

That is, locally, it has antiderivatives. The proof spans the several following pages and requires a few lemmas.

\subsection{The case of continuous derivative}

\begin{lemma}
  Suppose $f'$ is continuous. Then the form $f \d{z}$ is closed.
\end{lemma}

{While this might seem to give a hint of our further course of action, we won't be so blunt as to prove the continuity of $f'$ directly.}

\begin{idea}
  Indeed, as we know, the closedness is then equivalent to the equality
  $$ \frac{\pd f}{\pd y} = i\frac{\pd f}{\pd x}. $$
  This is straightforward to show using Cauchy-Riemann equations.
\end{idea}

\begin{example}
  \label{closed and not exact}
  The form ${\d z}/{z}$ is closed and not exact.
  $f' = -1/z^2$ in this case is a continuous function. By the previous lemma, $f$ is closed (in fact, its antiderivatives are logarithms). Now consider the unit circle $\S^1$. By parametrising with $e^{i\square}$, we can easily check that 
  $$ \int_{\S^1} \frac{\d z}{z} = 2 \pi i. $$
  But this means we have found a non-balanced region of $\C$, so $\d z / z$ is not exact.
\end{example}

\subsection{Indices}

\begin{lemma}
  Let $C = z_0 + r\S^1$ for some $r > 0$. Then
  \begin{align}
      \int_C \frac{\d z}{z - z_1} =
      \begin{cases}
        0, & \a{z_1-z_0} > r, \\
        2 \pi i, & \a{z_1-z_0} < r.
      \end{cases}
  \end{align}
\end{lemma}

\begin{proof}
  Consider the case $\a{z_1-z_0}>r$. Luckily, the form $\frac{\d z}{z - z_1}$ is closed in $H = \C \sm \b{z_1}$. This $H$ contains a rectangle around the square $C$. Every closed form is exact within this rectangle by a theorem from semester 3.
  
  Suppose, for now, $\a{z_1-z_0}<r$.
  We reduce this to the case $z_0 = z_1$, which has been considered in the example on page \pageref{closed and not exact}.
  Then
  \begin{align*}
    \int_C \frac{\d z}{z - z_1}
    &= \int_C \frac{\d z}{\p{z-z_0}-\p{z_1-z_0}} \\
    &= \int_C \frac{\d z}{z - z_0} \cdot \frac{1}{1 - \frac{z_1-z_0}{z-z_0}} \\
    &= \int_C \frac{\d z}{z-z_0} \cdot \sum_{k\in \N} \p{\frac{z_1-z_0}{z-z_0}}^k \\
    &= 2 \pi i + \sum_{k\in \N \sm 0} \int_C \frac{\d z}{z-z_0} \p{\frac{z_1-z_0}{z-z_0}}^k \\
    &= 2 \pi i.
  \end{align*}
\end{proof}

\subsection{The conclusion}

\begin{theorem}[Cauchy, on closedness]
  \label{Cauchy, on closedness}
  If $f \col G \to \C$ is holomorphic, then the form $f \d{z}$ is closed.
\end{theorem}

\begin{proof}
  Suppose otherwise: there exists a 2-dimensional rectangle $P \ss G$ with $I := \int_{\pd P} f \ne 0.$
  Subdivide it into four $\b{Q_i}$ such that
  $$ I = \sum_i \int_{\pd Q_i} f. $$
  For one of them, $Q_j$, the modulus of the integral is at least one fourth the $\a{I}$. Denote $P_1 = Q_j$. Continuing this sequence, we get diminishing $\b{P_j}$ with a single point $z_0$ in their intersection. The function $f$ is holomorphic at $z_0$, so we may write
  $$ f(z) = f(z_0)+f'(z_0)(z-z_0)+\phi(z), $$
  where $\phi(z) = o\p{z-z_0}$ with $z \to z_0$.
  Select $k$ such that $\a{\phi(z)} < \eps \a{z-z_0}$ for all $z \in P_k$.
  Then we have
  \begin{align*}
    \a{\frac{I}{4^k}}
    &\le\a{\ \int_{\pd P_k} f} \\
    &= \a{\ \int_{z \in \pd P_k} \p*[\big]{\ub{f(z_0)+f'(z_0)(z-z_0)}_{\text{these guys are exact in $P_k$}}\mathrel{+}\phi(z)}} \\
    &= \a{\ \int_{z \in \pd P_k} \phi(z)} \\
    &\le \eps \cdot \p{\Diam P_k} \cdot {S_1\p{\pd P_k}} \\
    &= \eps \cdot \frac{\p{\Diam P} \cdot S_1\p{\pd P}}{4^k}.
  \end{align*}
  But we thought that $I \ne 0$.
\end{proof}

\section{On correctible singularities}

\begin{lemma}[on correctible singularities]
  \label{correctible singularities}
  Let $a \in G$.
  Suppose the form $\go = f \d x + g \d y$ is closed in $G \sm \b{a}$, and the coefficients $f$ and $g$ are continuous in $G$. Then $\go$ is closed in $G$.
\end{lemma}

\begin{idea}
  Approximate integrals with smaller rectangles.
\end{idea}

\section{The minor integral formula of Cauchy}

\begin{theorem}[Cauchy, minor integral formula]
  \label{Cauchy, minor integral formula}
  Let $f \col G \to \C$ be holomorphic,
  $z_0, z_1 \in G$,
  $C = z_0 + r\S^1$,
  $\a{z_1 - z_0} < r$.
  Then
  $$ f(z_1) = \frac{1}{2 \pi i} \int_C \frac{f(z)}{z-z_1} \dd z. $$
\end{theorem}

`Probably the most important formula in complex analysis.' --- S.K.

The `greater' integral formula gives the same result for $C$ not necessarily a circle.

\begin{proof}
  Fix $z_0 \in G$.
  Consider
  $$
  h(z)
  =
  \begin{cases}
    \frac{f(z)-f(z_1)}{z-z_1}, & z \ne z_1 \\
    f'(z_1), & z = z_1.
  \end{cases}
  $$
  $h$ is continuous in $G$ and holomorphic in $G \sm \b{z_0}$.
  By Cauchy's theorem on closedness (page \pageref{Cauchy, on closedness}), the form $h \d z$ is closed in $G \sm \b{z_1}$.
  By the lemma on correctible singularities (page \pageref{correctible singularities}), it is also closed in all of $G$. Let $\wh C$ be a ball of slightly greater radius $r + \gd$, but still lying in $G$. Every closed form in $\wh C$ is exact, so
  $$ \int_C h \d z= 0. $$
  Rewriting this yields
  \begin{align*}
    f(z_1) \ub{\int_C \frac{\d z}{z-z_1}}_{=2\pi} = \int_C \frac{f(z)}{z - z_1} \d z.
  \end{align*}
\end{proof}

\section{Cauchy's theorem on analyticity}

\begin{theorem}[Cauchy, on analyticity]
  Let $f$ be holomorphic in $G$, $z_0 \in G$, $R = \dist\p{z_0, \pd G}$. 
  For all $k \in \N$, define
  $$ c_k = \frac{1}{2\pi i} \int_C \frac{f(z)}{\p{z-z_0}^{k+1}} \d z. $$
  Then
  $$ f(z) = \sum_{k \in \N} c_k \p{z-z_0}^k $$
  for all $z$ such that $\a{z-z_0} < R$.
  In particular, $f$ is analytic in $G$.
\end{theorem}

\begin{proof}
  Let $C$ be a circle around $z_0$ of radius $r < R$.
  By Cauchy's minor integral formula from page \pageref{Cauchy, minor integral formula}, for any $z$ inside of $C$ we have
  \begin{align*}
    f(z_1)
    &= \frac{1}{2\pi i} \int_C \frac{f(z)}{z-z_1} \d z \\
    &= \frac{1}{2\pi i} \int_C \frac{f(z)}{\p{z-z_0}-\p{z_1-z_0}} \d z \\
    &= \frac{1}{2\pi i} \int_C \frac{1}{z-z_0} \cdot \frac{f(z)}{1-\frac{z_1-z_0}{z-z_0}} \d z \\
    &= \sum_{k\in \N} \frac{1}{2\pi i} \int_C \frac{f(z)}{z-z_0} \cdot \p{\frac{z_1-z_0}{z-z_0}}^k \d z \\
    &= \sum_{k\in \N} \p{z_1-z_0}^k \cdot \p{ \frac{1}{2\pi i} \int_C \frac{f(z)}{\p{z-z_0}^{k+1}} \d z },
  \end{align*}
  which is desired.
  Transposing the sum with the integral is legit, since the series for geometric progression converges uniformly for all $z$.
\end{proof}

\begin{remark}
  In the analyticity theorem,
  $$ c_k = \frac{f^{(k)}(z_0)}{k!}. $$
  In particular,
  $$ f'(z) = \frac{1}{2\pi i} \int_{\pd B} \frac{f(\gz)}{\p{\gz-z}^2} \d \gz. $$
\end{remark}

\subsection{Morera's theorem}

\begin{corollary}[Morera's theorem]
  Let $f \col G \to \C$ be a continuous function.
  The following conditions are equivalent:
  \begin{enumerate}
    \item $f$ is analytic.
    \item $f$ is holomorphic.
    \item $f \d z$ is closed. 
  \end{enumerate}
\end{corollary}

%This, in particular, gives an alternative proof of the analyticity theorem.

\begin{proof}
  $1 \Rightarrow 2$ follows by differentiating the series, $2 \Rightarrow 3$ is the Cauchy's theorem on closedness from page \pageref{Cauchy, on closedness}. We now give a proof of $3 \Rightarrow 1$. By definition, the form $f \d z$ locally has an antiderivative $\Phi$. Then $\Phi$ is holomorphic, and $\Phi' = f$. By the analyticity theorem, $\Phi$ is holomorphic, which implies that $f$ is, too (simply differentiate the series). 
\end{proof}

\section{The mean value theorem}

\begin{lemma}[mean value theorem]
  If $C \ss G$ is a circle with the centre at $z_0 \in G$, then
  $$ f(z) = \frac{1}{2 \pi} \int_{C} f(z) \d z. $$
\end{lemma}

\begin{idea}
  Follows from the Cauchy's minor integral formula (page \pageref{Cauchy, minor integral formula}).
\end{idea}

\section{Maximum modulus principle}

\begin{theorem}[maximum modulus principle]
  Let $f \col G \to \C$ be holomorphic. Then $\a{f}$ has no strict maximum in $G$.
\end{theorem}

\begin{proof}
  Suppose $f$ has a non-strict maximum in $G$. Then
  \begin{align*}
    \a{f(a)}
    &= \frac{1}{2\pi} \a{\int_C f(z) \d z} \\
    &\le \frac{1}{2\pi} {\int_C \a{f(z)} \d z} \\
    &\le \a{f(a)}.
  \end{align*}
  Here every inequality must be an equality.
  Then $\a{f(z)} = \a{f(a)}$ for all $z$ in some disk $D$ around $a$.
  
  Suppose $f'(a) \ne 0$.
  Then $f$ is a local homeomorphism around $a$, which means it must map a ball in $\C$ into a ball in $\C$. But in any ball there are points with varying modulus.
  
  Therefore, $f$ is constant, what completes the proof.
\end{proof}

\section{Liouville's theorem}

\begin{theorem}[Liouville]
  A bounded entire function is constant.  
\end{theorem}

A powerful theorem.

\begin{proof}
  From principal estimate and the formula for $f'$:
  \begin{align*}
    \a{f'(z)}
    &\le \frac{M}{2\pi} \int_C \frac{r}{\p{r - \a{z}}} \d t \\
    &= \frac{Mr}{\p{r-\a{z}}^2} \\
    &\xleftarrow[r \to \infty]{} 0.
  \end{align*}
\end{proof}

\subsection{Principal theorem of algebra}

\begin{corollary}[principal theorem of algebra]
  $\C$ is algebraically closed.
\end{corollary}

\begin{proof}
  Suppose $p \col \C \to \C$ of $\deg p > 0$ has no roots: $\a{p} > \gd$ for some $\gd > 0$. Then $1/p$ is an entire function, and, as such, is either unbounded or constant. It is not constant, so it must be unbounded; but $1/\a{p} < 1/\gd$ --- a contradiction.
\end{proof}

\section{Harmonic functions}

\begin{definition}
  A function $f \col G \to \R$ is \emph{harmonic}, iff there exists a holomorphic $g \col G \to \C$ such that $\Re g = f$. If $\wt f = \Im g$, $f$ and $\wt f$ are said to be \emph{harmonic conjugates}.
\end{definition}

\begin{lemma}[Laplace's equations]
  If $u+iv \in \Hol G$, then
  $$ u'_{xx} + v'_{yy} = 0. $$
\end{lemma}

\begin{proof}
  From Cauchy-Riemann equations.
\end{proof}

\begin{definition}
  The differential operator
  $$ \Delta = \sum_{j=1}^n \p{\frac{\pd}{\pd x_j}}^2 $$
  is called the \emph{Laplace operator}.
\end{definition}

\begin{definition}
  A function $f \col H \to \R$ is called \emph{harmonic}, iff
  $$ \Delta f = 0. $$
\end{definition}

\begin{theorem}
  If $u \in C^2(G)$ and $\Delta u = 0$, then every $a \in G$ has a neighbourhood $U_a$ such that $u|_{U_a}$ is harmonic.
\end{theorem}

\begin{proof}
  Let $\phi = \frac{\pd u}{\pd x}$ and $\psi = -\frac{\pd u}{\pd y}$. We have
  $$ \frac{\pd \psi}{\pd x} = \frac{\pd \phi}{\pd y}, $$
  so there exists a function $v$ such that $\frac{\pd v}{\pd y} = \phi$ and $\frac{\pd v}{\pd x} = \psi$. This $v$ is harmonically conjugate to $u$, since the Cauchy-Riemann equations are satisfied.
\end{proof}


\section{Integrals of closed forms}

\begin{lemma}
  Suppose $\phi = \d \Phi$ in $G$. Then
  $$ \int_\gg \phi = \Phi(\gg(b)) - \Phi(\gg(a)) $$
  for every path $\gg \col [a, b] \to G$.
\end{lemma}

\begin{proof}
  By the theorem on exact forms.
\end{proof}

\begin{definition}
  A continuous function $F \col [a, b] \to \C$ is called the \emph{antiderivative along a path} $\gg \col [a, b] \to G$ of a 1-form $\phi$, iff for every $t \in [a, b]$ exists an antiderivative $\Phi$ in some neighbourhood $V_t \ni \gg(t)$ such that $F(t) = \Phi \circ \gg(t)$ for all $t \in V_t$.
\end{definition}

\begin{theorem}
  ~\begin{enumerate}
    \item The antiderivative $F$ exists.
    \item If $F_1$ and $F_2$ are two antiderivatives, there exists $c \in \C$ such that $F_1 = F_2 + c$.
  \end{enumerate}  
\end{theorem}

\begin{proof}[Proof of uniqueness]
  It follows from the definition that $F$ is locally constant.
\end{proof}

\begin{proof}[Proof of existence]
  Split $[a, b]$ into small subintervals (small enough for the form $\phi$ to have an antiderivative on the $\gg$-image of each). Choose constants in their intersections in such a way that they make an actual function (antiderivatives differ by a constant).
\end{proof}

\begin{remark}
  This can be used as an alternative definition of the integral along a path:
  $$ \int_\gg \phi := F(b) - F(a). $$
\end{remark}


\section{Homotopies}

\begin{definition}
  A \emph{homotopy} is a continuous map $h \col [\ga, \gb] \times [a, b] \to G$ (usually $\ga = 0$, $\gb = 1$).
\end{definition}

\begin{definition}
  A \emph{loop} is a closed path; i.e. a $\gg \col [a, b] \to \C$ that can be continued to an $(b-a)$-periodic one. A loop is \emph{contractible}, iff it is homotopic to a constant path.
  $G$ is \emph{simply-connected}, iff every loop in $G$ is contractible.
\end{definition}

\begin{example}
  All astroid domains are simply-connected.  
\end{example}

\begin{definition}
  Let $\phi$ be a 1-form in $G$, $S = [0, 1] \times [a, b]$, $h \col S \go G$ a homotopy.
  A continuous $F \col S \to G$ is an \emph{antiderivative of $\phi$ along $h$}, iff for every $(t, s) \in S$ there exists a ball $B \ss G$ with the centre at $h_t(s)$ and an antiderivative $U$ for $\phi$ in $B$.
\end{definition}

\begin{theorem}
  Such $F$ always exists in case of a closed $\phi$.
\end{theorem}

\begin{proof}
  $h(S)$ is compact. By closedness of  $\phi$, take balls such that there exists an antiderivative.
  By continuity of $h$ and Lebesgue's lemma, there exists $\gd > 0$ such that $h(e)$ lies completely in one of these balls for every $e$ of diameter less than $\gd$.
  Extract a finite subcover, and make them agree on intersections.
\end{proof}

\begin{theorem}
  Let $\gg_t$ be a homotopy in $G$, and all paths $\gg_t$ are closed.
  Let $\phi$ be a closed 1-form in $G$, then
  $$ \int_{\gg_0} \phi = \int_{\gg_1} \phi. $$  
\end{theorem}

\begin{proof}
  Define $g(t) = \int_{\gg_t} \phi.$ We assert this function is locally constant.
  Fix $t \in [a, b]$. Then
  $$ \int_{\gg_{t}} \phi = F\p{\gg_{t}(b)}-F\p{\gg_t(a)} \equiv \const $$
  for some antiderivative $F$ along $\gg_t.$ 
\end{proof}

\begin{corollary}
  In a simply-connected set, every closed form is exact.
\end{corollary}

\begin{proof}
  Because the integral over any closed path equals that over a constant one.
\end{proof}


\section{Laurent series}

\begin{definition}
  A \emph{Laurent series} is one of the form
  $$ \sum_{k \in \Z} c_k \p{z-z_0}^k $$
  for some $z, z_0, c_k \in \C$.
  It is said to \emph{converge} at $z$, iff both
  \begin{align*}
    \sum_{k \ge 0} &c_k \p{z-z_0}^k, &
    \sum_{k < 0} &c_k \p{z-z_0}^k
  \end{align*}
  converge at $z$. The first of these is the \emph{regular} part; the other one --- \emph{principal.}
\end{definition}

\begin{theorem}
  Let
  $$ A = \b{ z \in \C \mid r < \a{z-z_0}<R } $$
  (\emph{anneau})
  for some $0< r< R$. 
  If $f \col A \to \C$ is holomorphic, then $f$ decomposes uniquely into a Laurent series at any $z_0 \in A$.
\end{theorem}

\begin{idea}
  Consider integrals over the inner $c$ and outer $C$ components of the boundary of the ring $A$.
  For every $z \in Z$,
  $$ \int_c \frac{f(w)-f(z)}{w-z} \d w = \int_C \frac{f(w)-f(z)}{w-z} \d w. $$
  From this we get an expression for $2\pi i f(z)$ as a sum of integrals of two geometric series.
\end{idea}

\section{Singularities}

\begin{definition}
  Let $f \col G \sm \b{a} \to \C$ be holomorphic, $a \in G$.
  \begin{itemize}
    \item $a$ is \emph{removable}, iff $f$ is bounded in a neighbourhood of $a$.
    \item $a$ is a \emph{pole}, iff $f(z) \to \infty$ when $z \to a$.
    \item $a$ is \emph{essential}, iff it is not removable or a pole; i.e. $f$ has no limit at $a$.
  \end{itemize}
\end{definition}

\begin{lemma}
  Let $a$ be a correctible singularity of $f \col G \sm \b{a} \to \C$.
  Then $f$ can be continued to an analytic function $f \col G \to \C$.
\end{lemma}

\begin{proof}
  We assert every coefficient of the principal part is zero.
  Indeed,
  \begin{align*}
    c_k
    &= \frac{1}{2\pi i} \int_{a+rS^1} f(\gz) \p{\gz-a}^{-j-1} \d \gz, \\
    \\
    \a{c_k}
    &\le \frac{A \a{r}^z{-j-1} 2 \pi r}{2\pi} \\
    &\xrightarrow[r\to 0]{} 0. 
  \end{align*}
\end{proof}

\begin{lemma}
  $a$ is a pole iff
  the number of nonzero coefficients $c_k$ at $a$ with $k < 0$ is nonzero and finite.
  $a$ is essential, iff it is infinite.
\end{lemma}

\begin{idea}
  If $f \to \infty$, then $a$ is c removable for $1/f$.
\end{idea}